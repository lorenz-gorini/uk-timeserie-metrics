\documentclass[10pt]{article}
\usepackage[utf8]{inputenc}
\usepackage[T1]{fontenc}
\usepackage{amsmath}
\usepackage{amsfonts}
\usepackage{amssymb}
\linespread{1.2}
\usepackage{babel}
\usepackage{amsthm, bm}
\usepackage{graphicx}
\usepackage{cancel}
\usepackage{comment}
\usepackage{enumerate}
\usepackage{booktabs}
\usepackage{float}
\usepackage{subcaption}
\usepackage{geometry}
\geometry{a4paper,top=2.5cm,bottom=2.5cm,left=2cm,right=2cm}
\usepackage{caption}
\usepackage[version=4]{mhchem}
\usepackage{stmaryrd}
\usepackage{hyperref}
\usepackage{float}
\hypersetup{colorlinks=true, linkcolor=blue, filecolor=magenta, urlcolor=cyan,}
\urlstyle{same}
\usepackage{bbold}

\title{Timeserie Econometrics 3 - Problem Set 1}
\author{Alex Basov, Lorenzo Gorini}
\date{\today}

\begin{document}
	
	\maketitle
	\vspace{1cm}
	
	\section*{Lab Exercises}
	\subsection*{Question 1: VAR and impulse response functions (IRF)}
		Refer to the 3 code files for the solution. Particularly, the 2 python files are for data cleaning and merging. The R file "model.R" is to reply to the following questions.
		Here we only report the main results.
		
		\subsubsection*{Part 1}
		\textit{Before fitting a trivariate VAR(p) for the 3-variable you just collect, briefly discuss if there are any transformations needed for the raw data}
		We downloaded the 3 timeseries from UK Office of National Statistics:
		\begin{enumerate}
			\item "GDP at market prices: Current price: Seasonally adjusted £m" \\
			Data from: https://www.ons.gov.uk/economy/grossdomesticproductgdp/timeseries/ybha/ukea \\
			Source dataset ID: UKEA
			\item Inflation (CPIH INDEX 00- ALL ITEMS 2015=100)
			Data from:
			https://www.ons.gov.uk/economy/inflationandpriceindices/timeseries/l522/mm23
			Year base: 2015 -> set to 100
		\end{enumerate}
		Then, in order to merge them, we had to get the same periodicity for the data.
		Particularly,  
		
		The GDP data contain 2 timeseries:
		\begin{itemize}
			\item One is observed on a yearly basis
			\item One is observed on a quarterly basis
		\end{itemize}
		We want to keep only the quarterly data, which is in the format "YYYY QN" where N is
		the quarter number (1, 2, 3, or 4).
		
		In CPI data, there are actually 3 timeseries:
		\begin{itemize}
			\item One is observed on a yearly basis
			\item One is observed on a quarterly basis
			\item One is observed on a monthly basis
		\end{itemize}
		We want to keep only the quarterly data.
		
		Finally, since we want to run a SVAR model with the three time series, we need to assume weak stationarity. For this reason, we transform the cpi and gdp
		columns into inflation and gdp growth rate as the difference between the logarithm of the value of period t and the one of period t-1.
		
		\subsubsection*{Part 2}
		\textit{Estimate the model based on transformed series, show the estimates of the coefficients and of the covariance matrix you obtain}
		
		\subsubsection*{Part 3}
		\textit{Use the estimates above as DGP to design a Monte Carlo experiment. Assess the accuracy of the AIC and BIC criterion in selecting the lag order of a VAR. Comment on the results you obtain.}
		
		\subsubsection*{Part 4}
		\textit{Suppose now you want to identify and analyze the effects of a monetary policy shock:
			\begin{itemize}
				\item Comment on the ordering of the variables you would use to recursively identify the shock.
				\item Obtain IRFs estimates from both VAR and Local Projections method (Jorda (2005,AER)). Comment on the differences you find
			\end{itemize}
		}
		\newpage
%%%%%%%%%%%%%%%%%%%%%%%%%%%%%%%%%%%%%%%%%%%%%%%%%%%%	
% Change the style of enumeration
\renewcommand{\theenumi}{(\arabic{enumi})}
\renewcommand{\labelenumi}{(\arabic{enumi})}
%%%%%%%%%%%%%%%%%%%%%%%%%%%%%%%%%%%%%%%%%%%%%%%%%%%%	
		
		
		\section*{Theory Exercise}
		\subsubsection*{Question 4}
		
		
		\begin{enumerate}
			\item We know that $y_t$ is stationary, that is:
			\begin{itemize}
				\item $E[y_t] = \mu \quad \forall t$
				\item $E[y_t^2] < \infty \quad \forall t$
				\item $\gamma_j = E[(y_{t+j} - \mu)(y_t - \mu)] = E[(y_{h+j} - \mu)(y_{h} - \mu)] \quad \forall t, h, j$
			\end{itemize}
			\[
			\lim\limits_{T\to\infty} \gamma_j = 0 \implies \lim\limits_{T\to\infty} \frac{1}{T} \sum\limits_{j=0}^{T} \gamma_j = 0 \quad \text{(using Hint)}
			\]
			Prove that:
			\[
			\bar{y}_T \xrightarrow{p} \mu \iff \lim\limits_{T\to\infty} P(|\bar{y}_T - \mu| > \epsilon) = 0 \quad \forall \epsilon > 0
			\]
			Recall Chebyshev's inequality:
			\[
			\forall \epsilon > 0, \quad P(|\bar{y}_T - \mu| \ge \epsilon) \le \frac{Var(\bar{y}_T)}{\epsilon^2}
			\]
			\[
			Var(\bar{y}_T) = \frac{1}{T^2} Var\left(\sum\limits_{t=1}^{T} y_t\right) = \frac{1}{T^2} \left( \sum\limits_{t=1}^{T} Var(y_t) + 2 \sum\limits_{1 \le i < j \le T} Cov(y_i, y_j) \right) =
			\]
			
			\[
			= \frac{1}{T^2} \left( \sum\limits_{t=1}^{T} \gamma_0 + 2 \sum\limits_{t=1}^{T-1} \sum\limits_{j=1}^{T-t} \gamma_j \right) = \frac{1}{T^2} \left( T\gamma_0 + 2 \sum\limits_{j=0}^{T-1} (T-j)\gamma_j \right) =
			\]
			
			\[
			= \frac{1}{T^2} \left( T\gamma_0 + 2 T \sum\limits_{j=0}^{T-1} \gamma_j - 2 \sum\limits_{j=0}^{T-1} j\gamma_j \right) = \frac{\gamma_0}{T} + \frac{2}{T} \sum\limits_{j=0}^{T-1} \gamma_j - \frac{2}{T^2} \sum\limits_{j=0}^{T-1} j\gamma_j
			\]
			
			Now let's take the limit of the Chebyshev's inequality:
			\[
			\lim\limits_{T\to\infty} P(|\bar{y}_T - \mu| \ge \epsilon) \le \lim\limits_{T\to\infty} \frac{Var(\bar{y}_T)}{\epsilon^2} = \]
			\[= \frac{1}{\epsilon^2} \lim\limits_{T\to\infty} \left( \frac{\gamma_0}{T} + \frac{2}{T} \sum\limits_{j=0}^{T-1} \gamma_j - \frac{2}{T^2} \sum\limits_{j=0}^{T-1} j\gamma_j \right)
			\]
			
			Let's consider the limit on the RHS:
			\[
			\lim\limits_{T\to\infty} \left( \frac{\gamma_0}{T} + \frac{2}{T} \sum\limits_{j=0}^{T-1} \gamma_j - \frac{2}{T^2} \sum\limits_{j=0}^{T-1} j\gamma_j \right) = 0 + 2 \lim\limits_{T\to\infty} \frac{1}{T} \sum\limits_{j=0}^{T-1} \gamma_j - 2 \lim\limits_{T\to\infty} \frac{1}{T^2} \sum\limits_{j=0}^{T-1} j\gamma_j
			\]
			Note that
			\[
			\sum\limits_{j=0}^{T-1} j\gamma_j = 0\gamma_0 + 1\gamma_1 + 2\gamma_2 + 3\gamma_3 + \cdots + (T-1)\gamma_{T-1} = 
			\]
			\[
			= \sum\limits_{j=1}^{T-1} \gamma_j + \sum\limits_{j=2}^{T-1} \gamma_j + \sum\limits_{j=3}^{T-1} \gamma_j + \cdots + \sum\limits_{j=T-1}^{T-1} \gamma_j = \sum\limits_{j=1}^{T-1} \sum\limits_{i=j}^{T-1} \gamma_i
			\]
			
			Then the limit becomes:
			\[
			\lim\limits_{T\to\infty} \left( 2 \lim\limits_{T\to\infty} \frac{1}{T} \sum\limits_{j=0}^{T-1} \gamma_j - 2 \lim\limits_{T\to\infty} \frac{1}{T^2} \sum\limits_{j=1}^{T-1} \sum\limits_{i=j}^{T-1} \gamma_i \right) =
			= 2 \cdot 0 - 2 \sum\limits_{j=1}^{T-1} \frac{1}{T} \lim\limits_{T\to\infty}  \left( \frac{1}{T} \sum\limits_{i=j}^{T-1} \gamma_i \right) = 0
			\]
			\[
			\implies \lim\limits_{T\to\infty} P(|\bar{y}_T - \mu| \ge \epsilon) = 0 \quad \forall \epsilon > 0
			\]
			\[
			\implies \bar{y}_T \xrightarrow{p} \mu
			\]
			
			\item Show that $Cov(y_1, \bar{y}_T) = \dfrac{1}{T} \sum\limits_{s=0}^{T-1} \gamma_s$;
			\[
			Cov(y_1, \bar{y}_T) = Cov\left(y_1, \frac{1}{T} \sum\limits_{t=1}^{T} y_t\right) = \frac{1}{T} \sum\limits_{t=1}^{T} Cov(y_1, y_t)
			\]
			\[
			= \frac{1}{T} \sum\limits_{t=1}^{T} E[(y_1 - \mu)(y_t - \mu)] = \frac{1}{T} \sum\limits_{t=1}^{T} \gamma_{|t-1|} = \frac{1}{T} \left( \gamma_0 + \gamma_1 + \cdots + \gamma_{T-1} \right) = \frac{1}{T} \sum\limits_{s=0}^{T-1} \gamma_s
			\]
			
			\item
			Suppose that $\lim\limits_{T\to\infty} \sum\limits_{s=0}^{T} \gamma_s < \infty$, show that:
			\[
			\lim\limits_{T\to\infty} E[(\bar{y}_T - \mu)^2] = 0
			\]
			\[
			Var(\bar{y}_T) = \frac{\gamma_0}{T} + \frac{2}{T} \sum\limits_{j=0}^{T-1} \gamma_j - \frac{2}{T^2} \sum\limits_{j=0}^{T-1} j\gamma_j
			\]
			From derivations in (1):
			\[
			\lim\limits_{T\to\infty} Var(\bar{y}_T) = \lim\limits_{T\to\infty} \frac{\gamma_0}{T} + 2 \lim\limits_{T\to\infty} \frac{1}{T} \sum\limits_{j=0}^{T-1} \gamma_j - 2 \lim\limits_{T\to\infty} \frac{1}{T^2} \sum\limits_{j=0}^{T-1} j\gamma_j
			\]
			Using the hint, we observe that:
			\begin{itemize}
				\item First term clearly converges to 0
				\item In the second term: the sum converges to a constant while T in denominator goes to infinity. Hence the whole term converges to 0
				\item The third term is effectively the expression in the hint divided by T that goes to infinity. The whole term converges to 0
			\end{itemize}
			\[
			\lim\limits_{T\to\infty} Var(\bar{y}_T) = 0 + 2 \cdot 0 - 2 \cdot 0 = 0
			\]
			\[
			\lim\limits_{T\to\infty} E[(\bar{y}_T - \mu)^2] = \lim\limits_{T\to\infty} Var(\bar{y}_T) = 0
			\]
		\end{enumerate}
		
\end{document}
